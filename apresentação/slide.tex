\documentclass[xcolor=dvipsnames]{beamer}
\usepackage[brazil]{babel}
\usepackage[utf8]{inputenc}
\institute{IFPB - Instituto Federal da Paraíba}
\title{Quantidade de informação em Imagens }
\author{Arivanilton Junior, Felipe Lyra Barros e Barros e Jefferson Maximiliano Oliveira das Mercês}
\date{17/05/2018}
\usetheme{Copenhagen}
\begin{document}
\frame{\titlepage}

\frame{
\frametitle{Introdução}
A informação e a medida da probabilidade da ocorrencia, estando relacionada com a incerteza, o desconhecimento a priori.
Nesse trabalho, iremos explorar a aferição da quantidade de informação contida em uma imagem.}

\frame{
\frametitle{Linguagem utilizada}
Para o desenvolvimento dessa analise, foi utilizada a linguagem
de programação Python.Sua escolha se deu pelo fato de que essa linguagem possui uma enorme quantidade de bibliotecas voltadas a computação científica disponíveis a comunidade, como as utilizadas no projeto}

\frame{
\frametitle{Bibliotecas}
\begin{block}{Skimage}
A biblioteca scikit-image(ou skimage) e, em tradução livre da pagina do projeto, uma coleção de algoritmos para processamento de imagem, gratuita e livre de restrições, escrita por uma comunidade ativa de voluntarios.
\end{block}

\begin{block}{Matplotlib}
A biblioteca matplotlib e voltada a plotagem figuras 2D em uma grande variedade de formatos, podendo gerar graficos, histogramas, espectros de poténcia, grâficos de barras, graficos de erros, diagramas de dispersão, etc., com apenas algumas linhas de codigo.
\end{block}
}

\frame{
\frametitle{Código}
}


\end{document}