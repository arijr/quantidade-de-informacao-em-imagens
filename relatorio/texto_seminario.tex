\documentclass[journal]{IEEEtran}
\usepackage{blindtext}
\usepackage{graphicx}
\usepackage[utf8]{inputenc}
\usepackage[brazilian]{babel}

\usepackage{listings}
\usepackage{color}
 
\definecolor{dkgreen}{rgb}{0,0.6,0}
\definecolor{gray}{rgb}{0.5,0.5,0.5}
\definecolor{mauve}{rgb}{0.58,0,0.82}
 
\lstset{
  language=Python,                
  basicstyle=\footnotesize,           
  numbers=left,                   
  numberstyle=\tiny\color{gray},  
  stepnumber=2,                             
  numbersep=5pt,                  
  backgroundcolor=\color{white},    
  showspaces=false,               
  showstringspaces=false,         
  showtabs=false,                 
  frame=single,                   
  rulecolor=\color{black},        
  tabsize=2,                      
  captionpos=b,                   
  breaklines=true,                
  breakatwhitespace=false,        
  title=\lstname,                               
  keywordstyle=\color{blue},          
  commentstyle=\color{dkgreen},       
  stringstyle=\color{mauve},     
}
\usepackage{lipsum}


% correct bad hyphenation here
\hyphenation{op-tical net-works semi-conduc-tor}


\begin{document}

\title{Quantidade de informação em Imagens}


\author{Arivanilton Júnior, Felipe Lira e Jefferson Maximiliano\\

Instituto Federal de Educação, Ciência e Tecnologia da Paraíba\\Campus Campina Grande\\
E-mail:arivanilton.junior@mail.com; email2@mailxx.com e mail3@jjmail.com\\
}



% make the title area
\maketitle


\begin{abstract}
Escreva aqui o seu resumo. Blá blá... 
Blá blá...Blá blá...Blá blá...Blá blá...Blá blá...Blá blá...Blá blá...Blá blá...Blá blá...Blá blá...
Blá blá...Blá blá...Blá blá... Blá blá... Blá blá... Blá blá... Blá blá... Blá blá... Blá blá... Blá blá... Blá blá... 
Blá blá... Blá blá... Blá blá... Blá blá... Blá blá... Blá blá... Blá blá... Blá blá... Blá blá... Blá blá... Blá 
\end{abstract}



% Note that keywords are not normally used for peerreview papers.
\begin{IEEEkeywords}
Primeira Palavra-chave, Segunda Palavra-chave, Terceira Palavra-chave.
\end{IEEEkeywords}







\section{Introdução}

 A informação é a medida da probabilidade da ocorrência, estando relacionada com a incerteza, o desconhecimento a priori. Nesse trabalho, exploramos a aferição da quantidade de informação contida em uma imagem, descrevendo o comportamento do código proposto como possível solução...



\section{Descrição da solução}

\subsection{Linguagem}
    Para o desenvolvimento dessa analise, foi utilizada a linguagem de programação Python, que pode ser definida como:
\begin{quotation}Uma linguagem de programação de alto nível, interpretada, de script, imperativa, orientada a objetos, funcional, de tipagem dinâmica e forte.
\end{quotation}

Sua escolha se deu pelo fato de possuir uma enorme quantidade de bibliotecas voltadas a computação científica disponíveis a comunidade, como as utilizadas no projeto.
Dentre as varias distribuições do Python, optou-se por utilizar a Anaconda, uma distribuição Open Source e livre que possui além do Python, a linguagem R e mais de 250 pacotes voltados a data science, minimizando assim problemas com dependências de bibliotecas. 
\subsection{Bibliotecas}
No modelo trazido como solução, foram utilizadas as bibliotecas skimage e matplotlib.

\subsubsection{Skimage}

\begin{figure}[!ht]
	\centering
	\includegraphics[scale=0.3]{logo}
	\caption{logo Skimage}
\end{figure}
A biblioteca scikit-image(ou skimage) é, em tradução livre da pagina do projeto, uma coleção de algoritmos para processamento de imagem, gratuita e livre de restrições, escrita por uma comunidade ativa de voluntários. 

\subsubsection{Matplotlib}

\begin{figure}[!ht]
	\centering
	\includegraphics[scale=0.3]{logo2}
	\caption{logo matplotlib}
\end{figure}

A biblioteca matplotlib é voltada a plotagem figuras 2D em uma grande variedade de formatos, podendo gerar gráficos, histogramas, espectros de potência, gráficos de barras, gráficos de erros, diagramas de dispersão, etc., com apenas algumas linhas de código.

\subsection{Código}

 \begin{lstlisting}

import matplotlib.pyplot as plt
from skimage.io import imread
from skimage.filters.rank import entropy
from skimage.morphology import disk

imagem = "macacofezsilfie.jpg" 
image = imread(imagem, as_grey= True) 

fig, (ax0, ax1) = plt.subplots(ncols=2, figsize=(12, 4),sharex=True, sharey=True)

img0 = ax0.imshow(image, cmap=plt.cm.gray)
ax0.set_title("Imagem")
ax0.axis("off")
fig.colorbar(img0, ax=ax0)

img1 = ax1.imshow(entropy(image, disk(5)), cmap='gray')
ax1.set_title("Entropy")
ax1.axis("off")
fig.colorbar(img1, ax=ax1)

fig.tight_layout()

plt.show()

\end{lstlisting}
Veja como criar uma lista:
\begin{enumerate}
    \item Esse é o primeiro item
    \item O segundo item com uma citação~\cite{akyildiz2006next}.
\end{enumerate}


\begin{itemize}
    \item Esse é outro tipo de lista;
    \item Parece com o primeiro, mas não enumera;
    \item Pode colocar quantos quiser.
\end{itemize}


\section{Outra Seção}




\section{Conclusão}
Escreva aqui suas conclusões. \lipsum[1-2]

\section*{Agradecimentos}

Os autores deste trabalho agradecem ao IFPB, campus Campina Grande pelo apoio institucional. 

\bibliographystyle{IEEEtran}
\bibliography{referencias}

\end{document}


